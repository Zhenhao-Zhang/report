\documentclass{article}
\usepackage[UTF8]{ctex}
\usepackage{geometry}
\usepackage{natbib}
\geometry{left=3.18cm,right=3.18cm,top=2.54cm,bottom=2.54cm}
\usepackage{graphicx}
\pagestyle{plain}	
\usepackage{setspace}
\usepackage{caption2}
\usepackage{datetime} %日期
\renewcommand{\today}{\number\year 年 \number\month 月 \number\day 日}
\renewcommand{\captionlabelfont}{\small}
\renewcommand{\captionfont}{\small}
\begin{document}

\begin{figure}
    \centering
    \includegraphics[width=8cm]{upc.png}

    \label{figupc}
\end{figure}

	\begin{center}
		\quad \\
		\quad \\
		\heiti \fontsize{45}{17} \quad \quad \quad 
		\vskip 1.5cm
		\heiti \zihao{2} 《计算科学导论》课程总结报告
	\end{center}
	\vskip 2.0cm
		
	\begin{quotation}
% 	\begin{center}
		\doublespacing
		
        \zihao{4}\par\setlength\parindent{7em}
		\quad 

		学生姓名:\underline{\qquad  章震豪 \qquad \qquad}

		学\hspace{0.61cm} 号:\underline{\qquad 2014020228\qquad \quad} 
		
		专业班级:\underline{\qquad 计科2003 \qquad  \quad}
		
        学\hspace{0.61cm} 院:\underline{计算机科学与技术学院}
% 	\end{center}
		\vskip 2cm
		\centering
		\begin{table}[h]
            \centering 
            \zihao{4}
            \begin{tabular}{|c|c|c|c|c|c|c|}
            % 这里的rl 与表格对应可以看到,姓名是r,右对齐的;学号是l,左对齐的;若想居中,使用c关键字。
                \hline
                课程认识 & 问题思 考 & 格式规范  & IT工具  & Latex附加  & 总分 & 评阅教师 \\
                30\% & 30\% & 20\% & 20\% & 10\% &  &  \\
                \hline
                 & & & & & &\\
                & & & & & &\\
                \hline
            \end{tabular}
        \end{table}
		\vskip 2cm
		\today
	\end{quotation}

\thispagestyle{empty}
\newpage
\setcounter{page}{1}
% 在这之前是封面,在这之后是正文
\section{引言}
随着时代的发展,信息技术与计算机科学成为第四次科技革命的潮流。二十一世纪已成为信息爆炸化的时代,信息量呈几何性增长。计算机现已成为处理海量信息必不可少的工具。计算机技术应用范围不断扩大,包括航空运输、交通网络、专项科研、智能社区等方面。计算机技术发展对社会来说是一场革命,有着深远的意义。计算机科学也成为像数学,物理一样,一门必不可少的基础学科,包含了计算机各个领域中的不同内容。计算科学导论这么课程全面,系统的讲述了计算科学的架构,起源,发展与现状,并且带领我们展望未来。本文就计算科学导论认识的同时,叙述对软件工程能力培养的认识、计算机硬件系统的结构和工作过程以及个人所修习专业的看法。

\section{对计算科学导论这门课程的认识、体会}
出于对计算机科学的热爱,我转专业到了计算机科学与技术专业,并在大三学年修读了计算科学导论这门课程。虽然我已经在计算机专业度过了一年多的时光,但是上完这门课程之后,我再一次刷新了我对于计算科学与我自己专业的认识。\par
计算科学是一门很深的学科,在当今社会,这门学科也被运用到了生活中的方方面面。从身边来说,小时候家里打电话用的都是小灵通,到第一代3G手机有了屏幕,到第一代智能手机。对我自己来说,计算科学从一台用来打游戏,看动画片的机器逐渐到了融合百家之长的一门新兴,庞大的学科。从第一台计算机出现,到信息全球化,电脑走进每个人的家中,仅仅用了几十年的时间。计算科学更是社会发展进程中依赖比重很大的一门学科。在生活中更少不了它的使用。所以,学习计算机技术的重要性就显得尤为突出了。并且,大学生将会是未来社会中技术运用者的主力军。对于我们计算机技术水平和能力的要求,也提出了更高的挑战。\par
在我看来,我们这些学习计算机和技术的人至少应该具备以下能力:基本的计算机原理知识以及安装、维护、使用、设计和开发计算机应用程序的能力。具备开发平台的知识,掌握常用的开发工具,了解基本的软件开发方法。具有较强的数据库安装和故障排除能力以及简单的开发能力。获得信息管理系统的应用、开发和维护技术。具有设计、安装、排除故障和管理计算机网络的能力,掌握在计算机网络环境下开发计算机信息管理系统的基本方法和维护技能。当然,不同的学科符合不同的能力要求。


\par
对于一个初学者来说,我还不能胜任上述基本的计算机技能,但我相信在进一步的学习中,我将能把我的能力提高到更高的水平,并学到越来越多的计算机理论知识。和许多同学一样,我以前对计算机知之甚少,我开始从感性的角度理解计算机,一开始我只是对它有一个大致的印象,认为自己对计算机感兴趣。经过短短一学期的学习,我从理性的角度对计算机有了更深的了解。
\par
计算机科学和计算机技术是研究设计和建造计算机的理论、原理、方法和技术,以及使用计算机来获取、呈现、存储、处理和控制信息,包括科学和技术两个方面。科学是对现象和规律的研究,而技术是对计算机以及用于处理其信息的方法和技术的研究。因此,技术和科学是互补的,两者的结合是学科的基础。\par
谈到计算机科学和技术的研究,我们都知道,这两个领域不能简单地作为计算机科学和计算机技术单独研究或讨论--至少在本科阶段不能。因为计算机科学是以理论为基础的,它需要大量的实践来验证,而实践需要技术,两者相辅相成。计算机科学和计算机技术本质上是一门科学和工程学科,其特点是理论和实践的紧密结合。因此,我们学生不能把它分成科学和技术。在仍有时间和发展空间的情况下,先对该主题进行概述是一个好主意,以便为进一步的研究奠定基础。这之后是一个持续的浓缩期。在此期间,你将深入学习计算机知识,更多地了解系统原理、数据结构,最重要的是,在编程方面打下坚实的基础,培养你的基本技能,加强你的理论基础。在后期阶段,你将专注于发展你的创造性技能,将理论和技术应用于你的想法,让自己有机会实践并在实践中应用你所学的知识。归根结底,它是关于发展你的实践技能,并通过应用来理解计算机科学和技术。\par
根据自身的实际情况和结合自己的专业方向,我更感兴趣于机器学习,深度学习的方向。机器学习是大学学习的重要内容之一,而学习此门课程也得注意到一个问题,那就是机器学习,深度学习各领域的各种模型更新速度相当快。如何才能学好机器学习也是个需要解决的问题。以后的工作也很大可能性上是与机器学习有关。所以,在大学四年中,在修习其他学科的基础上,我将会将自己的研究方向朝向机器学习的领域。

\subsection{统计学习}
2021年9月,我在学习数学建模的过程中无意间了解到了统计学习这一理论。同年12月,在阅读了李航老师的《统计学习方法》\citep{tjxxff}后,我进一步对这个领域产生了好奇,我进一步阅读了统计学习常用模型:Lasso回归的一篇十分经典的论文《Regression Shrinkage and Selection via the Lasso》\citep{lasso},以及差异隐私相关的论文《Nearly-Optimal Private LASSO》\citep{dp}

统计学习技术是我们用来理解数据的工具,由万普尼克(Vapnik)建立的一套机器学习理论,使用统计的方法,因此有别于归纳学习等其它机器学习方法。由这套理论所引出的支持向量机对机器学习的理论界以及各个应用领域都有极大的贡献。这些工具可以分为有监督的或无监督(supervised or unsupervised)的。从广义上讲,监督统计学习涉及基于一个或多个输入来构建用于预测或估计输出的统计模型。这种性质的问题出现在商业,医学,天体物理学和公共政策等各个领域。在无监督的统计学习中,有输入但无监督输出。

统计学习理论是一种研究训练样本有限情况下的机器学习规律的学科。它可以看作是基于数据的机器学习问题的一个特例,即有限样本情况下的特例。
统计学习理论从一些观测(训练)样本出发,从而试图得到一些不能通过原理进行分析得到的规律,并利用这些规律来分析客观对象,从而可以利用规律来对未来的数据进行较为准确的预测。例如,对全国未来几年人口数量进行预测,就需要先采集到过去几年甚至几十年的人口数据,并对其变化规律做出统计学方面的分析和归纳,从而得到一个总体的预测模型,这样就可以对未来几年的人口总体走势作一个大概的估计和预测。
显然,这里采集到的过去人口的数据越准确,年份越长,分析归纳得到的统计规律就越准确,对未来人口预测就越接近真实水平。另外,如果只采集到了过去几年的人口数据,那么,这样得到的统计模型无论如何也是不够完美的。



\begin{center}
\includegraphics[width=1.0\linewidth]{tjxx}
\caption{图1:统计学习经典算法:多层感知机}
\end{center}

\par
统计学习方法包括模型的假设空间、模型选择的准则以及模型学习的算法,称为统计学习方法的三要素。

一般实现统计学习的步骤如下:

1.准备有限的训练数据集。

2.获取包含所有可能的模型的假设空间,即学习模型的集合。

3.确定模型选项的准则,即学习的策略。

4.实现最优模型的算法,即学习的算法。

5.通过学习方法选择最优模型。

6.利用学习的最优模型对新数据进行预则或分析。

统计学习依托背后的数学理论,在远早于机器学习大爆发的这十年,率先从解释因果的能力的角度,努力寻找上帝函数。的观点认为,统计学习里最重要的两个部分就是回归分析和假设检验。其他的方法或者技术在统计学习这个大框架下,最终也是为了这两者服务的。回归分析提供了解释因果的武器,假设检验则给这项武器装上了弹药。单纯的线性回归用最小二乘法求解逼近事实的真相,再使用显著性检验,检测变量的显著性、模型的显著性、模型的拟合精度。当然是否属于线性,也可以使用假设检验的方法检测。非线性回归的问题,使用极大似然估计或者偏最小二乘回归求解模型,后续的显著性检验仍然是一样的思路。显著性检验有它的局限性,这本身是由统计学习的一些限定假设引起的,在没有更强大的解释因果的方法框架出现前,它依然是解释因果的第一选择。虽然显得粗糙,但是能用。\par
从统计的角度逻辑回归可以得到严谨的数学解释和推断,全依赖于服从分布这个强假设。在这个假设下发展出的一整套理论,提供了现在这个通过数据学习世界的初级阶段,最优的解释因果框架。






\subsection{卷积神经网络}
2022年5月,偶然在B站看到了李沐老师的网课,7-8月自主阅读了《动手深度学习》\citep{dssdxx},了解到了卷积神经网络,并且进一步学习了经典的神经网络LeNet,AlexNet,VGG等。阅读了诸如《基于深度卷积神经网络的文档复原算法研究》\citep{jjsjwl}等实际应用的论文.

卷积神经网络(Convolutional Neural Networks, CNN)是一类包含卷积计算且具有深度结构的前馈神经网络(Feedforward Neural Networks),是深度学习(deep learning)的代表算法之一 。卷积神经网络具有表征学习(representation learning)能力,能够按其阶层结构对输入信息进行平移不变分类(shift-invariant classification),因此也被称为“平移不变人工神经网络(Shift-Invariant Artificial Neural Networks, SIANN)”   。
对卷积神经网络的研究始于二十世纪80至90年代,时间延迟网络和LeNet-5是最早出现的卷积神经网络  ;在二十一世纪后,随着深度学习理论的提出和数值计算设备的改进,卷积神经网络得到了快速发展,并被应用于计算机视觉、自然语言处理等领域  。
卷积神经网络仿造生物的视知觉(visual perception)机制构建,可以进行监督学习和非监督学习,其隐含层内的卷积核参数共享和层间连接的稀疏性使得卷积神经网络能够以较小的计算量对格点化(grid-like topology)特征,例如像素和音频进行学习、有稳定的效果且对数据没有额外的特征工程(feature engineering)要求

仍待解决的问题

首要的一点:开发使可视化评估更为客观的方法是非常重要的,可以通过引入评估所生成的可视化图像的质量和/或含义的指标来实现。\par
另外,尽管看起来以网络为中心的可视化方法更有前景(因为它们在生成可视化结果上不依赖网络自身),但似乎也有必要标准化它们的评估流程。一种可能的解决方案是使用一个基准来为同样条件下训练的网络生成可视化结果。这样的标准化方法反过来也能实现基于指标的评估,而不是当前的解释性的分析。\par
另一个发展方向是同时可视化多个单元以更好地理解处于研究中的表征的分布式方面,甚至同时还能遵循一种受控式方法。\par
使用共同的系统性组织的数据集,其中带有计算机视觉领域常见的不同难题(比如视角和光照变化),并且还必需有复杂度更大的类别(比如纹理、部件和目标上的复杂度)。事实上,近期已经出现了这样的数据集。在这样的数据集上使用 ablation study,加上对所得到的混淆矩阵的分析,可以确定 CNN 架构出错的模式,进而实现更好的理解。\par
此外,对多个协同的 ablation 对模型表现的影响方式的系统性研究是很受关注的。这样的研究应该能延伸我们对独立单元的工作方式的理解。\par
这些受控方法是很有前景的未来研究方向;因为相比于完全基于学习的方法,这些方法能让我们对这些系统的运算和表征有更深入的理解。这些有趣的研究方向包括:\par

逐步固定网络参数和分析对网络行为的影响。比如,一次固定一层的卷积核参数(基于当前已有的对该任务的先验知识),以分析所采用的核在每一层的适用性。这个渐进式的方法有望揭示学习的作用,而且也可用作最小化训练时间的初始化方法。\par
类似地,可以通过分析输入信号的性质(比如信号中的常见内容)来研究网络架构本身的设计(比如层的数量或每层中过滤器的数量)。这种方法有助于让架构达到适宜应用的复杂度。\par
最后,将受控方法用在网络实现上的同时可以对 CNN 的其它方面的作用进行系统性的研究,由于人们重点关注的所学习的参数,所以这方面得到的关注较少。比如,可以在大多数所学习的参数固定时,研究各种池化策略和残差连接的作用。
\begin{center}
\includegraphics[width=1.0\linewidth]{jjsjwl}
\caption{图2:卷积神经网络}
\end{center}

卷积神经网络工作过程:以卷积神经网络识别手写字体步骤为例

1、把手写字体图片转换成像素矩阵

2、对像素矩阵进行第一层卷积运算,生成六个feature map

3、对每个feature map进行下采样(也叫做池化),在保留feature map特征的同时缩小数据量。生成六个小图,这六个小图和上一层各自的feature map长得很像,但尺寸缩小了。

4、对六个小图进行第二层卷积运算,生成更多feature map

5、对第二次卷积生成的feature map进行下采样

6、第一层全连接层

7、第二层全连接层

8、高斯连接层,输出结果


\begin{table}[h]
    \centering
    \caption{几种常见卷积神经网络的对比}
\begin{tabular}{cccc}
% 这里的rl 与表格对应可以看到,姓名是r,右对齐的;学号是l,左对齐的;若想居中,使用c关键字。
    \hline
    名称 & 效率 &优点 & 其他\\
    \hline
    LeNet & 高&无&第一代神经网络 \\ 
    AlexNet &低&可以处理较大批量数据&第一代大型神经网络  \\
    VGG & 低&块神经网络,可以处理较难分类数据&块神经网络\\
    GoogLeNet&一般&处理图像较强&Image数据集冠军\\
    ResNet&一般&处理极端数据&残差神经网络\\
    \hline
\end{tabular}
    \label{table1}
\end{table}
\subsection{计算机视觉}
在学习卷积神经网络的过程中,逐步了解到了计算机视觉的相关内容,阅读了《人运动的视觉分析综述》\citep{jsjsj}来做为计算机视觉的启蒙。
计算机视觉是一门研究如何使机器“看”的科学,更进一步的说,就是是指用摄影机和电脑代替人眼对目标进行识别、跟踪和测量等机器视觉,并进一步做图形处理,使电脑处理成为更适合人眼观察或传送给仪器检测的图像。作为一个科学学科,计算机视觉研究相关的理论和技术,试图建立能够从图像或者多维数据中获取‘信息’的人工智能系统。这里所指的信息指Shannon定义的,可以用来帮助做一个“决定”的信息。因为感知可以看作是从感官信号中提 取信息,所以计算机视觉也可以看作是研究如何使人工系统从图像或多维数据中“感知”的科学。

计算机视觉发展局经历四个阶段



第一阶段是马尔计算视觉。1982年大卫马尔(David Marr)的《视觉》\citep{sj}一书在计算机视觉领域中起到了关键性的作用,它标志着计算机视觉正式成为一门独立的学科。马尔的计算视觉分为三个层次:计算理论、表达和算法以及算法实现。马尔认为,大脑的神经计算和计算机的数值计算没有本质区别,而从现在神经科学的进展看,“神经计算”与数值计算在有些情况下还是会产生本质区别。



主动视觉与目的视觉是第二阶段,学术界几位教授对马尔视觉计算理论提出了反对意见,认为缺乏主动性、目的性和应用性。但由于这段时期没有过多进展,对后续计算机视觉的发展影响不大,因此很多时候没有把这一阶段单独列出介绍。



多视几何和分层三维重建是计算机视觉发展的第三阶段,其中代表人物包括法国的O﹒Faugeras,澳大利亚国立大学的R﹒Hartely和英国牛津大学的A﹒Zisserman,在这方面的研究重点是如何快速、鲁棒地重建大场景。



最后来到了当代计算机视觉的阶段,基于深度学习的视觉。在此阶段中,文献大体上分为两个个阶段:以流形学习为代表的子空间法和目前以深度神经网络和深度学习为代表的视觉方法。

\begin{center}
\includegraphics[width=1.0\linewidth]{jsjsj}
\caption{图3:计算机视觉}
\end{center}

计算机视觉的2大挑战\par
对于人类来说看懂图片是一件很简单的事情,但是对于机器来说这是一个非常难的事情\par
\begin{itemize}
    \item 特征难以提取
    
    同一只猫在不同的角度,不同的光线,不同的动作下。像素差异是非常大的。就算是同一张照片,旋转90度后,其像素差异也非常大!所以图片里的内容相似甚至相同,但是在像素层面,其变化会非常大。这对于特征提取是一大挑战。
    
    \item 需要计算的数据量巨大
    
    手机上随便拍一张照片就是1000*2000像素的。每个像素 RGB 3个参数,一共有1000 X 2000 X 3=6,000,000。随便一张照片就要处理 600万 个参数.
    
\end{itemize}


计算机视觉的 8 大任务
\begin{itemize}
    
    \item 图像分类
    
    图像分类是计算机视觉中重要的基础问题。后面提到的其他任务也是以它为基础的。举几个典型的例子:人脸识别、图片鉴黄、相册根据人物自动分类等。

 
    \item 目标检测
    
    目标检测任务的目标是给定一张图像或是一个视频帧,让计算机找出其中所有目标的位置,并给出每个目标的具体类别。


    \item 语义分割
    
    它将整个图像分成像素组,然后对像素组进行标记和分类。语义分割试图在语义上理解图像中每个像素是什么(人、车、狗、树…)。除了识别人、道路、汽车、树木等之外,我们还必须确定每个物体的边界。
    
    
    \item 实例分割
    
    除了语义分割之外,实例分割将不同类型的实例进行分类,比如用5种不同颜色来标记5辆汽车。我们会看到多个重叠物体和不同背景的复杂景象,我们不仅需要将这些不同的对象进行分类,而且还要确定对象的边界、差异和彼此之间的关系。
    
    
    \item 视频分类
    
    与图像分类不同的是,分类的对象不再是静止的图像,而是一个由多帧图像构成的、包含语音数据、包含运动信息等的视频对象,因此理解视频需要获得更多的上下文信息,不仅要理解每帧图像是什么、包含什么,还需要结合不同帧,知道上下文的关联信息。
    
    
    \item 人体关键点检测
    
    人体关键点检测,通过人体关键节点的组合和追踪来识别人的运动和行为,对于描述人体姿态,预测人体行为至关重要。在 Xbox 中就有利用到这个技术。
    
    
    \item 场景文字识别
    
    很多照片中都有一些文字信息,这对理解图像有重要的作用。场景文字识别是在图像背景复杂、分辨率低下、字体多样、分布随意等情况下,将图像信息转化为文字序列的过程。停车场、收费站的车牌识别就是典型的应用场景。
    
    
    \item 目标跟踪
    
    目标跟踪,是指在特定场景跟踪某一个或多个特定感兴趣对象的过程。传统的应用就是视频和真实世界的交互,在检测到初始对象之后进行观察。无人驾驶里就会用到这个技术。
\end{itemize}

计算机视觉是一个跨学科的科学领域,涉及如何制作计算机以从数字图像或视频中获得高层次的理解。从工程的角度来看,它寻求自动化人类视觉系统可以完成的任务。

计算机视觉任务包括用于获取,处理,分析和理解数字图像的方法,以及从现实世界中提取高维数据以便例如以决策的形式产生数字或符号信息。

近几年来,计算机视觉技术应用场景愈加广泛,从中国产业信息网统计的数据显示,2017年计算机视觉行业市场规模中占比最高的是安防行业,占整个市场规模的七成。国内现在计算机视觉在安防领域相对比较成熟,“四小龙”为2017年安防领域表现最强的厂商,其他如云从、旷视均已大力发展安防市场。

\section{进一步的思考}
我们组演讲的内容是Lasso回归\par

关于Lasso回归的进一步思考:该方法是一种压缩估计。它通过构造一个惩罚函数得到一个较为精炼的模型,使得它压缩一些回归系数,即强制系数绝对值之和小于某个固定值;同时设定一些回归系数为零。因此保留了子集收缩的优点,是一种处理具有复共线性数据的有偏估计。Lasso是另一种数据降维方法,该方法不仅适用于线性情况,也适用于非线性情况。Lasso是基于惩罚方法对样本数据进行变量选择,通过对原本的系数进行压缩,将原本很小的系数直接压缩至0,从而将这部分系数所对应的变量视为非显著性变量,将不显著的变量直接舍弃。

LASSO回归的特点是在拟合广义线性模型的同时进行变量筛选(variable selection)和复杂度调整(regularization)。因此,不论因变量是连续的(continuous),还是二元或者多元离散的(discrete),都可以用 LASSO 回归建模然后预测。算法中的复杂度调整是指通过一系列参数控制模型的复杂度,从而避免过度拟合(overfitting)。

对于线性模型来说,复杂度与模型的变量数有直接关系,变量数越多,模型复杂度就越高。更多的变量在拟合时往往可以给出一个看似更好的模型,但是同时也面临过度拟合的危险。此时如果用全新的数据去验证模型(validation),通常效果很差。一般来说,变量数大于数据点数量很多,或者某一个离散变量有太多独特值时,都有可能过度拟合。

LASSO 回归复杂度调整的程度由参数$\lambda$ 来控制,λ越大对变量较多的线性模型的惩罚力度就越大,从而最终获得一个变量较少,而且比较有代表性的变量组合。

LASSO可以有效的避免过拟合,什么是过拟合,让我们看看下图,从左往右依次是欠拟合,拟合效果良好和过拟合。建模的过程就是模型对数据的普遍规律的总结,例如在线性模型中,输入自变量的数值,通过线性公式的转换,让最终得到的因变量的值和实际的因变量的值相差尽量小,这就是模型的拟合过程。欠拟合情况下,模型没有正确认识到数据体现的普遍规律,无法对作出良好的预测。

过拟合情况下,模型将数据中的全部信息都当做普遍规律,在训练集样本可以达到完美的效果,但用于样本外的测试情况就不行了,因为它可能将一些片面的、只体现于小部分样本的规律误认为是复合总体样本的普遍规律了,这种的模型最终效果也是不好的,我们要避免这两种情况,

一般来说,欠拟合可以用复杂的算法来避免,但随着模型复杂度增大,其学习率也增强,很容易出现过拟合现象。LASSO回归加入了正则项,对不蕴含有用信息的特征进行减小权重的操作,从而达到减小过拟合的目的。

以二维数据空间为例,说明lasso和Ridge两种方法的差异,左图对应于Lasso方法,右图对应于Ridge方法。
如上图所示,两个图是对应于两种方法的等高线与约束域。红色的椭圆代表的是随着λλ的变化所得到的残差平方和,βˆβ^为椭圆的中心点,为对应普通线性模型的最小二乘估计。左右两个图的区别在于约束域,即对应的蓝色区域。
等高线和约束域的切点就是目标函数的最优解,Ridge方法对应的约束域是圆,其切点只会存在于圆周上,不会与坐标轴相切,则在任一维度上的取值都不为0,因此没有稀疏;对于Lasso方法,其约束域是正方形,会存在与坐标轴的切点,使得部分维度特征权重为0,因此很容易产生稀疏的结果。
所以,Lasso方法可以达到变量选择的效果,将不显著的变量系数压缩至0,而Ridge方法虽然也对原本的系数进行了一定程度的压缩,但是任一系数都不会压缩至0,最终模型保留了所有的变量。


问题回答:
\begin{itemize}
    \item Lasso回归为什么要有惩罚项? 
    
    为了解决过拟合问题,并降低估计量的方差(若存在严格多重共线性,则 OLS 方差无穷大),常使用“惩罚回归”(penalized regression),即在 OLS的目标函数(残差平方和)之外,再加上一个惩罚项,惩罚太大的回归系数。使得系数矩阵趋向于Lq范数最小化,达到变量选择的目的
    \item 为什么不用于高次数据,只用于线性数据?
    
    最基本的Lasso回归在线性数据上的作用效果会更好,对于高次数据,也可以类比多层感知机来构造多层Lasso,并且在层与层之间施加合适的激活函数来起到传递的效果。
    
    \item Lasso在回归中的定位,作用范围是什么?
    
    Lasso回归是一种"年轻"的回归,最早提出来医学病理统计领域,用来筛选出影响病人病情的主要原因,适用范围较为狭窄。现在Lasso主要作用于统计学习,差分隐私,联邦学习,高维数据降维,以及医学方面的问题上。
\end{itemize}


\section{总结}
新时代是IT行业的蓬勃发展时代,作为即将走出社会的我们应该抓住时代发展的主旋律,在此行业中探寻自身发展的道路。计算机科学如同参天大树,里面的分支犹如它的根茎庞大而复杂。经过这门课程的学习,使得我初步认识了计算机这一学科,它是深奥的,但同时也是有着巨大作用的一门学科。希望在以后的学习和工作当中能接触到更深层次的知识。\par
对于自己的个人能力,希望之后的学习能够扎实系统的学习计算机理论专业知识,通过更高层次的理论学习与项目实战来提高自己的视野与能力,解决更多的问题,人生的路才刚刚开始,还有很多地方需要去探索,去发现,还有很多地方需要去努力实现。\par



\section{附录}
\begin{itemize}
    \item 申请Github账户,给出个人网址和个人网站截图
    
    个人网址https://github.com/Zhenhao-Zhang
    
    个人网站截图
    \begin{center}
    \centering
    \includegraphics[width=1.0\linewidth]{github.png}
    \caption{图4:Github主页}
 
    \end{center}
    
    \item 注册观察者、学习强国、哔哩哔哩APP,给出对应的截图
    
    观察者
    
    \begin{center}
    \centering
    \includegraphics[scale=0.1]{gcz}\par
    \caption{图5:观察者主页}
    \end{center}
    
    学习强国
    
    \begin{center}
    \centering
    \includegraphics[scale=0.1]{xxqg}\par
    \par
    \caption{图6:学习强国主页}
    \end{center}
    
    哔哩哔哩
    \begin{center}
    \centering
    \includegraphics[scale=0.1]{bilibili}\par
    \caption{图7:哔哩哔哩主页}
    \end{center}
    \item 注册CSDN、博客园账户,给出个人网址和个人网站截图
    
    CSDN个人网址https://blog.csdn.net/zgsdzzh
    
    CSDN个人主页截图
    \begin{center}
    \centering
    \includegraphics[width=1.0\linewidth]{csdn}
    \caption{图8:CSDN主页}
    \end{center}
    
    博客园个人网址https://home.cnblogs.com/u/3030237
    
    博客园个人主页截图
    \begin{center}
    \centering
    \includegraphics[width=1.0\linewidth]{bky}
    \caption{图9:博客园主页}
    \end{center}
    
    \item 注册小木虫账户,给出个人网址和个人网站截图
    
    个人网址http://muchong.com/bbs/space.php?uid=32276824
    
    个人网站截图
    \begin{center}
    \centering
    \includegraphics[width=1.0\linewidth]{xmc}
    \caption{图10:小木虫主页}
    \end{center}
    
\end{itemize}


\hspace*{\fill} \\


\bibliographystyle{unsrt}
\bibliography{references}


\end{document}
