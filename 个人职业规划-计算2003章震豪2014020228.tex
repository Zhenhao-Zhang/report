\documentclass{article}
\usepackage[UTF8]{ctex}
\usepackage{geometry}
\usepackage{multirow}
\usepackage{natbib}
\geometry{left=3.18cm,right=3.18cm,top=2.54cm,bottom=2.54cm}
\usepackage{graphicx}
\pagestyle{plain}	
\usepackage{setspace}
\usepackage{enumerate}
\usepackage{caption2}
\usepackage{datetime} %日期
\renewcommand{\today}{\number\year 年 \number\month 月 \number\day 日}
\renewcommand{\captionlabelfont}{\small}
\renewcommand{\captionfont}{\small}
\begin{document}

\begin{figure}
    \centering
    \includegraphics[width=8cm]{upc.png}

    \label{figupc}
\end{figure}

	\begin{center}
		\quad \\
		\quad \\
		\heiti \fontsize{45}{17} \quad \quad \quad 
		\vskip 1.5cm
		\heiti \zihao{2} 《计算科学导论》个人职业规划
	\end{center}
	\vskip 2.0cm
		
	\begin{quotation}
% 	\begin{center}
		\doublespacing
		
        \zihao{4}\par\setlength\parindent{7em}
		\quad 

		学生姓名:\underline{\qquad  章震豪 \qquad \qquad }

		学\hspace{0.61cm} 号:\underline{\qquad 2014020228\qquad\quad}
		
		专业班级:\underline{\qquad 计科2003 \qquad  \quad}
		
        学\hspace{0.61cm} 院:\underline{计算机科学与技术学院}
% 	\end{center}
		\vskip 1.5cm
		\centering
		\begin{table}[h]
            \centering 
            \zihao{4}
            \begin{tabular}{|c|c|c|c|c|c|c|c|c|}
            % 这里的rl 与表格对应可以看到,姓名是r,右对齐的;学号是l,左对齐的;若想居中,使用c关键字。
                \hline
                \multicolumn{5}{|c|}{分项评价} &\multicolumn{2}{c|}{整体评价}  & 总    分 & 评 阅 教 师\\
                \hline
                自我 & 环境 & 职业 & 实施 & 评估与 & 完整性 & 可行性 &\multirow{2}*{} &\multirow{2}*{}\\
                分析& 分析& 定位 & 方案 & 调整 & 20\% & 20\% & ~&~ \\\            
                10\% & 10\% & 15\% & 15\% & 10\% & &  &~ &~\\
                \cline{1-7} 
                & & & & & & & ~&~ \\
                & & & & & & & ~&~ \\
                \hline      
            \end{tabular}
        \end{table}
		\vskip 2cm
		\today
	\end{quotation}

\thispagestyle{empty}
\newpage
\setcounter{page}{1}
% 在这之前是封面,在这之后是正文
\section{自我分析}
	首先,我将从多个角度对我个人的情况进行一些评价与分析,以便认识自我,对职业做出正确的选择,从而选定适合自己发展的职业生涯路线。
\subsection{自然条件}
\begin{itemize}
\item 性别:男
\item 年龄:20
\item 健康状况:身体状态健康,精神状态良好,可以保持长时间的专注与工作。
\item 居住城市: 湖北省鄂州市,武汉8+1城市圈,内陆小县城,经济与科技发展较为落后。

\end{itemize}
\subsection{性格分析}
\begin{itemize}
\item 霍兰德职业兴趣测试

\begin{center}
    \centering
    \includegraphics[width=1.0\linewidth]{hld}
    \caption{霍兰德测试结果}
 
    \end{center}
    
霍兰德测试中与我相关的代码是:E,C,S,符合我在日常生活中注重交际,乐意与能够给我帮助的人为伍,同时事业心很强,十分喜欢竞争与碰撞,好胜心强

\item MBTI测试

\begin{center}
    \centering
    \includegraphics[width=1.0\linewidth]{MBTI}
    \caption{MBTI测试结果}
 
    \end{center}

MBTI测试中我的代码是ENTP--外向,逻辑,直觉,展望。与我的人格是“辩论家”,符合我在生活,学习上喜欢对某一个观点报有自己的看法,且不会轻易向某个观点妥协,同时,较有创造性与想象力,相信自己的判断。

综上所述,我的性格较为外向活泼,思想丰富且对于自己的观点较为固执,有些暴躁,有一定的创造性与想象力,能够为一个观点坚持思考很久,处理事情一定要找到一定的理论依据。
\end{itemize}
\par
\subsection{教育与学习经历}
\begin{itemize}
\item 小学就读于全市最好的小学鄂州市东方红小学尖子班
\item 中学就读于市四大名校鄂州市第八中学尖子班,提高尖子班
\item 高中就读于全国重点高中,湖北省教育示范窗口华中师大一附中高分班
\item 大学一年级就读于985工程优势学科创新平台,211工程,双一流工程大学中国石油大学(华东)材料科学与工程学院
\item 大二转专业至计算机科学与技术学院学习
\end{itemize}
\par
\subsection{工作与社会阅历}
\begin{itemize}
\item 思群软件有限公司--后端开发运维\par
大一寒假凭借着自己大一上学期学习的部分开发知识进入思群软件公司“历练”,没有承接具体的开发任务,但是感受到了互联网公司与企业的工作环境与工作氛围,也学会了诸如centos7的操作,nginx,Tomcat的使用等
\item 华为云第七期高校青年班--java后端开发\par
大一下学期参加了华为云高校青年班,用两周时间开发了一个简单的商城项目,虽然只用到了简单的Mysql,Springboot和Vue,但是这也是我第一次独立完成一个项目
\item 百度松果菁英班--深度学习,计算机视觉\par
大二学年加入了百度松果菁英班,接触了机器学习与深度学习的知识,开始接触计算机视觉相关的内容,完成了一些小的计算机视觉项目

综上所述,我在大学过程中积极参加各类技术研究,开发活动,努力学习知识,也收到了诸如美团,知乎等大厂的招聘电子邮件的橄榄枝,以及阿里,字节,腾讯,pjlab等企业学长的内推邀请,也希望最后能够给自己的大学生活画上一个完整的句号。

\end{itemize}

\par
\subsection{知识、技能与经验}
\begin{itemize}
\item 知识\par
首先,拥有良好的数学功底,高考数学142分,数学竞赛初赛全国三等奖,数学建模美赛特等奖提名(学校认定为全国一等奖),良好的数学能力是学好计算机的基础,也是处理各类问题的关键。\par
其次,拥有较为扎实的计算机功底,在我所选择的相关方向:机器学习与深度学习上,所有相关课程均取得90+的成绩,坚实的基础也是后续进行研究的基石。\par
最后,学习了较多的深度学习相关知识,正在阅读李航老师《统计学习方法》与李沐老师《动手深度学习》,掌握更多的机器学习,深度学习知识。
\item 技能\par
首先,熟悉后端开发,熟练使用轻量级后端框架Django,能够熟练操作Ubuntu 18.04服务器,配置Mysql,nginx,tmux等\par
其次,熟悉深度学习相关模型与项目流程,诸如GoogleNet,VGG,xgboost等,熟悉计算机视觉,能独立完成计算机视觉相关项目
\item 经验\par
首先,开发过多个后端项目,第一作者取得软件著作权三项\par
其次,进行过多个数据分析,数据建模,计算机视觉相关项目,能够进行一些算法上的设计与探索
\end{itemize}
\par
\subsection{兴趣爱好与特长}
\begin{itemize}


\item 兴趣爱好\par
思考,喜欢去想一些别人觉得没有意义的东西,去漫无边际的想一些东西。喜欢开发一些没有什么意义的小demo来供给自己娱乐,有时候也喜欢做一些花里胡哨的东西
\item 特长\par
接受新鲜事物的速度极快,可以很快的从0到做出一个基本的模型或框架
。

\end{itemize}
\subsection{性格分析}
\begin{itemize}
\end{itemize}
\par
\section{环境分析}
环境分析主要是评估周边各种环境因素对自己职业生涯发展的影响。每一个人都处在一定的环境之中,职业发展必然要受到所处环境的影响,只有充分了解和把握所处环境的现状、特点、发展变化趋势,才能做到在复杂的环境中避害趋利,使你的职业生涯规划具有实际意义。\par
环境分析包括:\par
\subsection{社会环境分析}
\begin{itemize}
\item 政治形势\par
从国内看,现如今,中国正走在发展道路上的关键点,在未来的30年,将是中国建成社会主义现代化强国的决胜时期。如今中国政治形势稳定,在中国共产党的领导下,国家凝聚力、人民团结程度达到新高度。在可见的未来,中国都将坚持社会主义,坚持中国共产党的领导,稳定的政治环境是经济发展的保障。\par
从国际看,联合国五常中俄美英法仍然起举足轻重的作用,地区组织发挥一定的影响力。联合国五常中俄美英法仍然起举足轻重的作用,地区组织发挥一定的影响力。国际政治形势在变局中深刻发展,新冠疫情给世界各国带来巨大冲击,逆全球化和单边主义继续酝酿,大国竞争明显升温,当今世界正处于百年未有之大变局。
\item 经济形势\par
“中国经济是一片大海,而不是一个小池塘;狂风骤雨可以掀翻小池塘,但不能掀翻大海;经历了无数次狂风骤雨,大海依旧在那儿!”另外,我国经济已经进入工业化晚期,城乡水平迅速拉平。IT 产业是世界第一大产业,未来 IT 产业将改变世界所有产业。中美欧主导国际经济总趋势市场消费降低,居民生活必需品销售稳中有增,居民就业和增收压力上升,基本民生保障力度加大。脱贫攻坚难度加大,三大攻坚战扎实推进,外部风险挑战上升,外贸外资基本盘大体稳定。2020年上半年,新冠疫情对我国经济发展和世界政经格局造成重大冲击。随着我国对疫情的有效防控,二季度实现大部分复工复产,消费、投资、工业企业利润等的降幅均出现不同程度收窄,经济呈修复企稳态势,但仍受国内部分地区疫情反复拖累。同时,受全球疫情蔓延及世界变局的影响,疫情不仅对中国对外贸易增速形成拖累,也导致全球产业链和供应链重新调整及贸易保护主义叠加,加上全球性、地域性政经摩擦和冲突导致的不确定性急剧上升,进一步加剧了经济下行压力。
\item 就业形势\par
在就业形势方面,最近几年较为严峻,而热门的互联网行业发展放缓,就业岗位减少。无疑在未来的几年内,IT行业的就业难度将会提高。但就业市场对高素质人才的需求量仍然难以满足,虽然互联网行业人才略有饱和,但基础行业仍需要大量人才。人工智能的热度在放缓,但考察到最近中国的发展战略,华为遭到美国禁运,为了真正突破国外的科技封锁,未来在集成电路、微电子等行业中将会投入大量资金,同时需要大量人才。但同时,由于互联网红利即将过去,整个IT行业的就业形势仍将变得严峻。
\end{itemize}
\subsection{学校环境}
\begin{itemize}
\item 学校总体环境\par
中国石油大学(华东)是教育部直属全国重点大学,是国家“211工程”重点建设和开展“985工程优势学科创新平台”建设并建有研究生院的高校之一。学校还是教育部和五大能源企业集团公司、教育部和山东省人民政府共建的高校,是石油石化高层次人才培养的重要基地,被誉为“石油科技、管理人才的摇篮”,现已成为一所以工为主、石油石化特色鲜明、多学科协调发展的大学。2017年、2022年均进入国家“双一流”建设高校行列。

追溯学校历史,1953年新中国成立之初,国民经济建设急需石油资源,石油工业发展急需专业人才。在这种形势下,以清华大学石油工程系为基础,汇聚北京大学、天津大学、大连工学院等著名高校的相关师资力量和办学条件,组建成立了新中国第一所石油高等学府——北京石油学院,隶属燃料工业部,是当时北京著名的八大学院之一。1960年10月,学校被确定为全国重点高校。1969年,学校迁至胜利油田所在地——山东东营,更名为华东石油学院。1981年6月在北京石油学院原校址内成立研究生部。1988年,学校更名为石油大学,逐步形成山东、北京两地办学格局。1997年,石油大学正式进入国家“211工程”首批重点建设高校行列。2000年,石油大学由中国石油天然气集团公司划归教育部。2000年6月,经教育部批准,学校成立研究生院。2003年10月,教育部与国家四大石油公司签署共建石油大学协议。2004年8月,教育部批准石油大学(华东)立项建设青岛校区。2005年1月,学校更名为中国石油大学。2005年8月,教育部与山东省人民政府签署共建中国石油大学(华东)协议。2006年10月,学校以“优秀”成绩通过教育部本科教学工作水平评估。2010年,学校成为国家首批实施“卓越工程师教育培养计划”的61所试点高校之一和承担国家“专业学位研究生教育综合改革试点工作”的32家部属高校之一。2014年4月,教育部与中国石油天然气集团公司、中国石油化工集团公司、中国海洋石油总公司、神华集团有限责任公司、陕西延长石油(集团)有限责任公司等五大能源企业集团公司签署共建中国石油大学协议。2018年12月,教育部、山东省人民政府重点共建中国石油大学(华东)。2021年,经教育部批准,东营校区调整为东营科教园区,办学主校区调整到青岛。

学校总占地面积5000余亩,建筑面积130余万平方米,发展形成了“两校区一园区”(青岛唐岛湾校区、古镇口校区以及东营科教园区)的办学格局。青岛两校区地处迷人的帆船之都、海滨之城,享有极高美誉的青岛,东营科教园区地处黄河三角洲的中心城市、生态之城、石油之城——山东东营,“两校区一园区”均位于“蓝黄”两大国家战略重点区域,同时青岛两校区还处于2014年新设立的国家级新区——青岛西海岸新区。学校建有研究生院,有地球科学与技术学院,石油工程学院,化学化工学院,机电工程学院,储运与建筑工程学院,材料科学与工程学院,石大山能新能源学院,海洋与空间信息学院,控制科学与工程学院,青岛软件学院、计算机科学与技术学院,理学院,经济管理学院,外国语学院,文法学院,马克思主义学院,体育教学部等16个教学学院(部),以及荟萃学院、国际教育学院、远程教育学院和继续教育学院。

学科专业覆盖石油石化工业的各个领域,石油主干学科总体水平处于国内领先地位。有14个博士学位授权一级学科,3个博士学位授权自主设置二级学科,9个博士授权自主设置交叉学科,2种博士专业学位授权类别,33个硕士学位授权一级学科,1个硕士学位授权二级学科,15种硕士专业学位授权类别,59个本科招生专业,11个博士后流动站。拥有矿产普查与勘探、油气井工程、油气田开发工程、化学工艺、油气储运工程等5个国家重点学科,以及地球探测与信息技术、工业催化等2个国家重点(培育)学科。工程学、化学、材料科学、地球科学、计算机科学、环境与生态学、社会科学总论等7个学科领域进入ESI全球学科排名前1%,其中工程学进入ESI全球学科排名前1‰。地质资源与地质工程、石油与天然气工程2个一级学科入选国家“双一流”建设计划。地质资源与地质工程、石油与天然气工程、化学工程与技术、安全科学与工程、地质学、地球物理学等6个一级学科进入教育部第四轮学科评估全国前十名。

学校教育体系完备,各类教育层次结构合理,现有全日制在校本科生近18900人、研究生9400余人,留学生970余人。建校以来,学校始终坚持以人才培养为根本任务,着力打造人才培养质量品牌,赢得了广泛的社会声誉。从广大校友中涌现出大批杰出人才,走出了30位两院院士以及一大批石油石化行业领军人物和工程技术骨干。自国家实行毕业生双向选择就业政策以来,毕业生就业率连续27年保持在90%以上,2004年被国务院授予“全国就业先进工作单位”荣誉称号,2011年入选50所全国毕业生就业典型经验高校,2019年入选“国家创新人才培养示范基地”。

学校拥有一支师德高尚、业务精湛、结构合理、充满活力的高素质教师队伍。现有教师1700余人,其中教授、副教授1100余人,博士生导师329人。有两院院士8人(含双聘),国家“万人计划”科技创新领军人才6人,长江学者特聘教授3人,国家杰出青年科学基金获得者6人,“973计划”项目首席科学家1人,国家“百千万人才工程”入选者13人;中青年科技创新领军人才6人,长江学者青年学者5人,国家优秀青年科学基金获得者6人,“新世纪优秀人才支持计划”入选者22人,中国青年科技奖获得者4人,教育部高校青年教师奖、霍英东教育基金会青年教师基金及青年教师奖获得者11人;泰山学者攀登计划专家5人,泰山学者特聘专家15人,泰山学者青年专家25人;山东省有突出贡献的中青年专家18人;山东省自然科学杰出青年基金获得者15人。国家“万人计划”教学名师2人,国家级教学名师奖获得者1人,省级教学名师14人,全国模范教师4人,全国优秀教师2人。国家自然科学基金创新研究群体1个,教育部创新团队3个,山东省优秀创新团队2个,泰山学者优势特色学科人才团队1个;全国高校黄大年式教师团队2个,国家级教学团队3个,山东省高校黄大年式教师团队2个,山东省教学团队11个。

学校是石油石化行业科学研究的重要基地,在基础理论研究、应用研究等方面具有较强实力,在10多个研究领域居国内领先水平和国际先进水平。现有重质油国家重点实验室、海洋物探及勘探开发装备国家工程研究中心、非常规油气开发教育部重点实验室、油气加工新技术教育部工程研究中心、石油石化新型装备与技术教育部工程研究中心等34个国家及省部级科研平台。

学校坚持开放办学,不断拓展社会服务领域和发展空间,与国内60多家地方政府、大型企事业单位签署了全面合作协议。学校重视国际交流与合作,已与美国、法国、加拿大、澳大利亚、英国、俄罗斯等45个国家和地区的200余所高等院校和学术机构建立了实质合作交流关系。聘请了百余名著名专家、知名人士为我校兼职教授、名誉教授和客座教授。近年来,国际合作交流项目逐步增加,呈现出良好的发展前景。

建校近70年来,学校形成了鲜明的办学特色,办学实力和办学水平不断提高。在新的历史时期,学校坚持特色发展、内涵发展、高质量发展,正向着“中国特色能源领域世界一流大学”的办学目标奋力迈进。
\item 学院环境\par
青岛软件学院、计算机科学与技术学院其前身为创建于1984年的计算机科学系,2001年撤系建院,成立计算机与通信工程学院,2019年院部调整更名为计算机科学与技术学院。经过近40年的不懈努力,已发展成为石油石化行业领先、国内具有较高知名度的IT科技创新和人才培养基地。2020年,为深入贯彻国家软件发展战略,满足国家能源安全战略及青岛市工业互联网之都建设对高端人才的需求,中国石油大学(华东)与青岛市政府、中国石油集团东方地球物理勘探有限责任公司、东软集团有限公司、海信集团有限公司共建青岛软件学院,并于2021年12月获批国家特色化示范性软件学院,从此开启崭新的发展阶段。

学科发展迅速,计算机科学进入ESI全球排名第340名(2022年7月),排名前5.3‰。设有先进科学与工程计算、油气人工智能2个交叉学科博士点,计算机技术与资源信息工程二级博士点,计算机科学与技术、软件工程2个硕士学位授权一级学科,拥有资源信息工程博士专业学位授权领域,计算机技术、软件工程2个硕士专业学位授权领域。其中,计算机应用技术为山东省重点学科。学院建有山东省能源工业大数据发展创新实验室、青岛市随钻仪器及信息处理工程技术研究中心、青岛市油气物联网与人工智能技术工程研究中心、青岛市大型石油工业软件工程研究中心,海洋物探及勘探设备国家工程实验室综合环境数字化与模拟分实验室等多层次、体系化的高端科研实验平台。形成智能信息处理、网络与服务计算、图形图像与可视化、数据科学与信息系统、软件工程理论、软件工程技术和油气领域软件服务工程等特色鲜明的学科方向,积累了大量优秀科研成果。十三五期间,获得教育部科技进步二等奖2项、山东省科技进步二等奖1项,中国计算机学会科技进步杰出奖1项,其他省部级科研奖励5项,在服务于国家能源战略和山东省、青岛市区域经济发展方面发挥了重要作用。

拥有一支高学历、高水平、经验丰富的师资队伍。现有教职工110余人,其中教授18人,博士生导师13人,副教授44人,具有博士学位人员57人,一年以上出国访学经历教师29人。有国家级青年拔尖人才1人,泰山领军人才1人,泰山青年学者1人,山东省青年科技奖2人,青岛市青年科技奖1人,青岛西海岸新区高层次领军人才2人,青岛西海岸新区高层次拔尖人才1人,青岛西海岸新区优秀青年人才2人。

近五年,承担各类纵、横向科研项目239项,其中主持承担国家重点研发计划项目等国家级科研项目30余项,省部级项目70余项,科研项目总经费达到5000余万元。在国内外重要学术期刊和学术会议上发表学术论文600余篇,其中SCI收录200余篇、EI收录300余篇,ESI高被引论文9篇;申报发明专利202件,授权发明专利42项,软件著作权112件。学院注重产学研相结合,与国内外著名石油石化企业密切合作,开发了一系列具有完全自主知识产权的石油领域软件平台和嵌入式设备,在国内外石油石化行业具有较大影响。

人才培养体系完整,设有计算机科学与技术、软件工程、物联网工程、智能科学与技术四个本科专业。其中,计算机科学与技术专业入选国家级一流专业建设点,为山东省特色专业;软件工程专业入选国家级一流本科专业建设点,为省“卓越工程师教育培养计划”建设专业。以立德树人为根本,通过持续综合改革不断提升人才培养质量。建院以来为国家输送了7100余名(其中,本科生5800余名,硕士生1300余名)具有较高科学素养、较强创新意识和实践能力的复合型高级工程技术人才。

近五年,学院本科和硕士毕业生初次就业率均位居学校各专业前列,就业层次显著提升。目前在读本科学生1400余名,研究生400余名。学院建有计算机网络、软件工程、物联网工程等专业实验室18个,大学生创新实验室8个。实验室占地面积2700平方米,固定资产2960万元,教学设备2580台套。学院重视与国内外著名企业合作加强人才协同培养,近五年获批教育部产学合作协同育人项目40余项,与科大讯飞、IBM、西门子、浪潮、思科、微软、朗讯、中软、东软等数十家企业开展了全方位合作。

 “格物计算,致知力行”。学院正致力于打造特色化的发展和育人文化,打造“三支队伍”,落实科教、产教和学科三个融合,建设“五大平台”,持续探索信息与计算的科学世界,培养高端人才,服务国家战略需求、行业与区域发展需要,推动社会发展。

\end{itemize}
\subsection{家庭环境分析}
\begin{itemize}
\item 婚姻状况\par
父母婚姻状况稳定,感情良好
\item 经济状况\par
父母均为体制内人员,父亲为教师,母亲为医生,家庭收入一般,生活水平较为稳定。
\item 家人期望\par
父母期望可以顺利毕业并且推免或考取研究生,将来找到一份好的工作,自食其力。
\item 家族传统\par
无特殊家族传统,仅有自食其力,遵纪守法的家规。
\end{itemize}
\subsection{职业环境分析}
\begin{itemize}

\item 行业现状及发展趋势\par
现如今计算机行业的发展十分迅速,计算机行业已成为当今最为火热的行业之一,在可见的未来里,信息技术行业仍将高速前进,对人才保持高需求。计算机的从业人员现已工作在许多行业与岗位,分支众多,职业种类多。不考虑科研界的话,总体可分为硬件行业与软件行业。硬件行业的工作内容包括集成电路的设计,计算机体系结构设计等,要求从业者拥有扎实的电子技术以及计算机底层基础,同时一般也需要精通较为底层的高级编程语言如C语言。难度较高。发展前景广阔,尤其是在中国当今大力振兴芯片行业的环境下,硬件行业将迎来一波新的机遇。软件行业如果不考虑管理岗、美术岗等小众选择,大量计算机从业人员都成为了程序员,主要工作内容是按照要求为公司开发大型程序,又可分为算法岗和业务岗,工作一般要求掌握良好的软件工程能力,要求精通给定的高级语言技术栈,要求有良好的逻辑思维能力与算法能力,要求有较强的持续学习能力。
\item 职业的工作内容与要求\par
对项目经理负责,负责软件项目的详细设计、编码和内部测试的组织实施,对程序员小
型软件项目兼任系统分析工作,完成分配项目的实施和技术支持工作;协助项目经理和相关人员同客户进行沟通,保持良好的客户关系;参与需求调研、项目可行性分析、技术可行性分析和需求分析;熟悉并熟练掌握交付软件部开发的软件项目的相关软件技术;负责向项目经理及时反馈软件开发中的情况,并根据实际情况提出改进建议;参与软件开发和维护过程中重大技术问题的解决,参与软件首次安装调试、数据割接、用户培训和项目推广;负责相关技术文档的拟订;负责对业务领域内的技术发展动态进行分析研究。
\item 职业的发展前景\par
发展前景在前几年因移动互联网的爆发和人工智能的兴起而变得炙手可热,一度形成了一切都转互联网的场景,近几年有所放缓,但对高级人才的需求量仍难以满足.在我国,IT产业在过去5年经历了年28\%的增长速度,是同期国家GDP增长速度的三倍,对GDP增长的拉动作用已进一步增强,对我国国民经济增长的贡献率不断提高。信息化产业的发展,带动IT业的发展,IT业的发展急需更多人才,网络工程师就是IT业中不可或缺的人才之一。总体来说是一个向上的朝阳行业,就业难一直都是热门的话题,但是计算机行业的从业者在就业时却有着很大的优势,因此只要通过自己的努力,不断的进取掌握专业知识,未来也能够有一个很好的发展前途。
\end{itemize}
行业现状及发展趋势;职业的工作内容、工作要求、发展前景等。\par


\subsection{地域与人际环境分析}
计划前往深圳工作,希望能够进入腾讯,阿里等大厂。
\begin{itemize}
    \item 气候水土\par
    深圳是中国南部海滨城市,毗邻香港。位于北回归线以南,东经113°46′至114°37′,北纬22°27′至22°52′。地处广东省南部,珠江口东岸,东临大亚湾和大鹏湾;西濒珠江口和伶仃洋;南边深圳河与香港相连;北部与东莞、惠州两城市接壤。辽阔海域连接南海及太平洋。深圳属亚热带季风气候,长夏短冬,气候温和,日照充足,雨量充沛。年平均气温23.3℃,历史极端最高气温38.7℃,历史极端最低气温0.2℃;一年中1月平均气温最低,平均为15.7℃,7月平均气温最高,平均为29.0℃;年日照时数平均为1853.0小时;年降水量平均为1932.9 毫米,全年86\%的雨量出现在汛期(4~9月)。春季天气多变,常出现“乍暖乍冷”的天气,盛行偏东风;夏季长达6个多月(平均夏季长202天),盛行偏南风,高温多雨;秋冬季节盛行东北季风,天气干燥少雨。深圳气候资源丰富,太阳能资源、热量资源、降水资源均居全省前列,但又是灾害性天气多发区,春季常有低温阴雨、强对流、春旱等,少数年份还可出现寒潮;夏季受锋面低槽、热带气旋、季风云团等天气系统的影响,暴雨、雷暴、台风多发;秋季多秋高气爽的晴好天气,是旅游度假的最好季节,但由于雨水少,蒸发大,常有秋旱发生,一些年份还会出现台风和寒潮;冬季雨水稀少,大多数年份都会出现秋冬连旱,寒潮、低温霜冻也是这个季节的主要灾害性天气。
    \item 文化特点\par
    深圳经济特区四十年发展,不仅创造了举世瞩目的经济奇迹,还实现了文化的崛起。2008年被联合国教科文组织授予“设计之都”,2009年被世界知识城市峰会授予“杰出的发展中的知识城市”,2013年被联合国教科文组织授予“全球全民阅读典范城市”,连续四次获评“全国文化体制改革先进地区”,连续五次荣获“全国文明城市”称号……一步一个脚印,深圳文化发展正是一个不断认识自己、实现自己、超越自己的过程,也是中国当代文化自信自觉自强的很好例证。
    \item 发展前景\par
    深圳是全国重要交通枢纽,有一些独特的地理优势,经济也呈现出稳固的增长趋势,深圳也是一个快节奏的城市,是强者的天堂,弱者的地狱。中国目前最著名的互联网城市都在一线城市,包括北京、上海、深圳、杭州、广州等。可以预测在将来这些城市仍将飞速发展壮大,因为城市化是国家发展所注定的结果,未来将有更多的人口迁徙到大城市。一线城市一般生活便利,文化底蕴深厚,业余生活丰富,生活惬意,且薪酬较高。但同时生活成本较高,房价高昂,人际关系薄弱,一个人易体会到孤独感,压力较大,不适合心理脆弱人群。
    \item 人际与人脉关系\par
    较多的高中同学在珠三角区域深造,较多学长学姐在深圳各大互联网企业工作,都是可以利用的人脉资源。
    
\end{itemize}


\section{职业定位}

\subsection{行业领域定位与理由}
经过对自身和环境的分析,我计划未来从事深度学习算法工程师,主要从事计算机视觉相关工作。\par
首先,从自身来说,我擅长使用Python语言进行编程,并且学习了大量机器学习,深度学习相关的知识,做过一些计算机视觉相关的项目,对这个领域较为熟悉且充满热情,立志在这个领域做出一定成绩。\par
其次,从大环境来说,第三次人工智能浪潮,深度学习助力感知智能步入成熟。不断提高的计算机算力加速了人工智能技术的迭代,也推动感知智能进入成熟阶段,AI与多个应用场景结合落地、产业焕发新生机。2006年深度学习算法的提出、2012年AlexNet在ImageNet训练集上图像识别精度取得重大突破,直接推升了新一轮人工智能发展的浪潮。2016年,AlphaGo打败围棋职业选手后人工智能再次收获了空前的关注度。从技术发展角度来看,前两次浪潮中人工智能逻辑推理能力不断增强、运算智能逐渐成熟,智能能力由运算向感知方向拓展。目前语音识别、语音合成、机器翻译等感知技术的能力都已经逼近人类智能。而计算机视觉作为深度学习最热门的领域,被应用于各个领域,也有很大的上升空间。
\par
\subsection{职业岗位起点定位与理由}
职业岗位起点为:计算机视觉算法工程师\par
当前深度学习领域正在高速发展的阶段,目前已经有一定的深度学习基础。通过本科生,研究生的学习,掌握卷积神经网络,计算机视觉相关的理论,学会设计,改进计算机视觉相关算法,并且能将算法用来应用。
\par
\subsection{职业目标与可行性分析}
\begin{itemize}
    \item 成果目标\par
    参与,主导一项计算机视觉框架的设计与开发
    \item 经济目标\par
    年薪40万及以上
    \item 能力目标\par
    掌握机器学习,深度学习基本理论,精通计算机视觉模型的推导与应用,能够自己修改,开发出计算机视觉框架
    \item 职务目标\par
    算法架构师
\end{itemize}
\par
成果目标、经济目标、能力目标、职务目标等。\par 
\begin{enumerate}[(1)]
	\item 短期目标(大学4年)
	\begin{itemize}
    \item 大一学年(基本完成)\par
    大一熟练掌握python,高等数学,打好理科基础,学习算法,参加算法竞赛(最后获得了蓝桥杯省级一等奖,进入全国决赛),转专业至计算机科学与技术(成功)。
    \item 大二学年(基本完成)\par
    参与大创项目(以及通过两次审核),学好线性代数,数据分析,数据结构等,参与大数据,理论机器学习相关项目比赛,参加数学建模竞赛(获得全国一等奖)。
    \item 大三学年\par
    学好专业课,提升自身能力,争取保研,多做计算机视觉相关项目
    \item 大四学年\par
    前往大厂实习,提升自我能力,顺利完成毕业设计
\end{itemize}
	\item 中长期目标(5-10年)。
	顺利进入硕士研究生阶段,在这一阶段在研究上有所突破和创新,有可能的话,希望可以读博士。
	参与计算机视觉相关研究,发表高质量论文。
	参与计算机视觉相关项目,锻炼项目能力
	
\end{enumerate}



\section{实施方案}

\begin{enumerate}[1、]
	\item 如何利用现有条件和自身优势以实现职业生涯目标。\par
	目前自己正处于大三阶段,,我们最需要的是三种能力,一是数学的能力,二是英语的能力,三是编程的能力,要做到自律,尽量减少娱乐时间,将更多的时间放在学习上。同时利用我深度学习,计算机视觉知识较为丰富的优势,多做项目,完善自我。
	\item 如何克服缺点、弥补不足、增长知识、提高能力以实现职业生涯目标。\par
	在当前,我的知识水平仍有不足,有待提升学习的深度不够,要不断正视差距,正视自己,脚踏实地,努力学习。同时,我目前的工程能力一般,参与的项目很少,在毕业前的时间里我将把自己的重心放在项目水平上,努力多接触一些科研包括工程项目,为研究生乃至毕业后的职业生涯做好准备。
	\item 如何处理人际关系和发展人脉以实现职业生涯目标。\par
	多参加文体活动、讲座报告、创新竞赛和社会实践等,可以遇到志同道合的人,向优秀的
人汲取经验;遇到问题主动向学长学姐请教,获得知识的同时发展了人脉;有任务积极主
动承担,学会推销自己,尽快让他人对自己留下印象。朋友较多,在认识的人里人缘较好。
且高中朋友分布于全国各地各个行业中,人脉积累较好,在以后将成为一大优势。而我也将努
力改变性格,多认识一些朋友,处理好人际关系,既能在生活中带来快乐,也可能在工作中带
来机会。
	\item 如何处理工作与家庭、生活的关系以实现职业生涯目标。\par
	我的家庭十分开明自由,父母会尊重我的每一步选择,但是也会提出他们的供我参考,我会综合两者,做出最优的判断。
	\item 如何处理释放工作压力、保证身心健康以实现职业生涯目标。\par
	应当努力把握好放松和工作之间的平衡。同时,热爱文学与音乐,在学习之余大量阅读经典文学作品与聆听古典音乐,大大缓解了我的心理压力。我喜欢运动,包括游泳,滑冰等。运动可同时提高人的身心素质,让我有体力,有强健的体魄来为社会主义建设事业添砖加瓦。
\end{enumerate}
\par 


\section{评估与调整}
每学期评估一次
\subsection{评估内容}
在大学四年里,评估内容应为学业成绩及能力目标,评估自己是否学会了期待的知识,是
否完成了项目,是否参与了计划参加的比赛。比如这一学年计划完成雅思考试,是否达到了预
计的分数。同时每一年应评估科技发展的形势,既然选择了新兴行业作为目标,那么在毕业之
前就应不断评价此行业的发展前景,及时动态做出调整或更改目标。在本科毕业时,应进行一
次总体评估,根据升学的结果来评估执行结果并做出调整。比如如果没有选到想要的方向,没
有去到期望的学校,计划应该做出什么样的改变。在研究生时,评价自己的能力,是否达到了
预计的要求。在工作后,每年评估自身能力,评估收入状况,评估发展前景,评估心理健康状
况,评估是否达到了想要的职务。同时评估行业发展情况,在看不到前途时及时退出。
\subsection{调整原则}
若能力目标没有达到,应根据原因进行调整,若是因时间问题,应该调整工作学习娱乐方
式,为提升能力留出更多时间,若是因个人智力水平问题,应降低能力目标。如果行业发展
前景脱离了计划,应及时调整行业,评价自身现有能力,为转向其它方向做出准备,比如就算
工业互联网已没有前景,自身掌握的软件开发水平应仍能满足于软件开发行业的需求,不管是
转向互联网还是算法岗都应做好准备。在学校时,就因根据行业形势,在评估时不断调整自己
的学习重心,信息技术行业的泡沫太多,一旦走错了方向,代价是高昂的,应同时为多个备选
方案作准备,调整自己的学习目标。在难以适应工作环境时,应考虑改变工作,由于我个人对
物质追求较低,环境适应力较好,故无需太过于考虑收入方面的问题,如果兴趣发生了转移,
就该及时调整现有环境。如果总体目标未能达到,比如未能保研,应及时把目标调整至出国深
造,若调整后还未来得及通过语言考试,应考虑调整计划实施时期,留出一年时间进行缓冲。
若心理健康状况发生严重问题,应作重大调整,将重心转移到个人生活上来,解决情绪问题。
若国家政策发生变化,应及时再次评估行业前景,努力往国家需要的地方上靠,努力为社会主
义现代化事业贡献出自己的一份力量。




\end{document}
